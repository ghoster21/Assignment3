\documentclass[]{article}
\usepackage{lmodern}
\usepackage{amssymb,amsmath}
\usepackage{ifxetex,ifluatex}
\usepackage{fixltx2e} % provides \textsubscript
\ifnum 0\ifxetex 1\fi\ifluatex 1\fi=0 % if pdftex
  \usepackage[T1]{fontenc}
  \usepackage[utf8]{inputenc}
\else % if luatex or xelatex
  \ifxetex
    \usepackage{mathspec}
  \else
    \usepackage{fontspec}
  \fi
  \defaultfontfeatures{Ligatures=TeX,Scale=MatchLowercase}
\fi
% use upquote if available, for straight quotes in verbatim environments
\IfFileExists{upquote.sty}{\usepackage{upquote}}{}
% use microtype if available
\IfFileExists{microtype.sty}{%
\usepackage{microtype}
\UseMicrotypeSet[protrusion]{basicmath} % disable protrusion for tt fonts
}{}
\usepackage[margin=1in]{geometry}
\usepackage{hyperref}
\hypersetup{unicode=true,
            pdftitle={Assignment3\_BigData},
            pdfauthor={Agus Setiyawan},
            pdfborder={0 0 0},
            breaklinks=true}
\urlstyle{same}  % don't use monospace font for urls
\usepackage{color}
\usepackage{fancyvrb}
\newcommand{\VerbBar}{|}
\newcommand{\VERB}{\Verb[commandchars=\\\{\}]}
\DefineVerbatimEnvironment{Highlighting}{Verbatim}{commandchars=\\\{\}}
% Add ',fontsize=\small' for more characters per line
\usepackage{framed}
\definecolor{shadecolor}{RGB}{248,248,248}
\newenvironment{Shaded}{\begin{snugshade}}{\end{snugshade}}
\newcommand{\KeywordTok}[1]{\textcolor[rgb]{0.13,0.29,0.53}{\textbf{#1}}}
\newcommand{\DataTypeTok}[1]{\textcolor[rgb]{0.13,0.29,0.53}{#1}}
\newcommand{\DecValTok}[1]{\textcolor[rgb]{0.00,0.00,0.81}{#1}}
\newcommand{\BaseNTok}[1]{\textcolor[rgb]{0.00,0.00,0.81}{#1}}
\newcommand{\FloatTok}[1]{\textcolor[rgb]{0.00,0.00,0.81}{#1}}
\newcommand{\ConstantTok}[1]{\textcolor[rgb]{0.00,0.00,0.00}{#1}}
\newcommand{\CharTok}[1]{\textcolor[rgb]{0.31,0.60,0.02}{#1}}
\newcommand{\SpecialCharTok}[1]{\textcolor[rgb]{0.00,0.00,0.00}{#1}}
\newcommand{\StringTok}[1]{\textcolor[rgb]{0.31,0.60,0.02}{#1}}
\newcommand{\VerbatimStringTok}[1]{\textcolor[rgb]{0.31,0.60,0.02}{#1}}
\newcommand{\SpecialStringTok}[1]{\textcolor[rgb]{0.31,0.60,0.02}{#1}}
\newcommand{\ImportTok}[1]{#1}
\newcommand{\CommentTok}[1]{\textcolor[rgb]{0.56,0.35,0.01}{\textit{#1}}}
\newcommand{\DocumentationTok}[1]{\textcolor[rgb]{0.56,0.35,0.01}{\textbf{\textit{#1}}}}
\newcommand{\AnnotationTok}[1]{\textcolor[rgb]{0.56,0.35,0.01}{\textbf{\textit{#1}}}}
\newcommand{\CommentVarTok}[1]{\textcolor[rgb]{0.56,0.35,0.01}{\textbf{\textit{#1}}}}
\newcommand{\OtherTok}[1]{\textcolor[rgb]{0.56,0.35,0.01}{#1}}
\newcommand{\FunctionTok}[1]{\textcolor[rgb]{0.00,0.00,0.00}{#1}}
\newcommand{\VariableTok}[1]{\textcolor[rgb]{0.00,0.00,0.00}{#1}}
\newcommand{\ControlFlowTok}[1]{\textcolor[rgb]{0.13,0.29,0.53}{\textbf{#1}}}
\newcommand{\OperatorTok}[1]{\textcolor[rgb]{0.81,0.36,0.00}{\textbf{#1}}}
\newcommand{\BuiltInTok}[1]{#1}
\newcommand{\ExtensionTok}[1]{#1}
\newcommand{\PreprocessorTok}[1]{\textcolor[rgb]{0.56,0.35,0.01}{\textit{#1}}}
\newcommand{\AttributeTok}[1]{\textcolor[rgb]{0.77,0.63,0.00}{#1}}
\newcommand{\RegionMarkerTok}[1]{#1}
\newcommand{\InformationTok}[1]{\textcolor[rgb]{0.56,0.35,0.01}{\textbf{\textit{#1}}}}
\newcommand{\WarningTok}[1]{\textcolor[rgb]{0.56,0.35,0.01}{\textbf{\textit{#1}}}}
\newcommand{\AlertTok}[1]{\textcolor[rgb]{0.94,0.16,0.16}{#1}}
\newcommand{\ErrorTok}[1]{\textcolor[rgb]{0.64,0.00,0.00}{\textbf{#1}}}
\newcommand{\NormalTok}[1]{#1}
\usepackage{graphicx,grffile}
\makeatletter
\def\maxwidth{\ifdim\Gin@nat@width>\linewidth\linewidth\else\Gin@nat@width\fi}
\def\maxheight{\ifdim\Gin@nat@height>\textheight\textheight\else\Gin@nat@height\fi}
\makeatother
% Scale images if necessary, so that they will not overflow the page
% margins by default, and it is still possible to overwrite the defaults
% using explicit options in \includegraphics[width, height, ...]{}
\setkeys{Gin}{width=\maxwidth,height=\maxheight,keepaspectratio}
\IfFileExists{parskip.sty}{%
\usepackage{parskip}
}{% else
\setlength{\parindent}{0pt}
\setlength{\parskip}{6pt plus 2pt minus 1pt}
}
\setlength{\emergencystretch}{3em}  % prevent overfull lines
\providecommand{\tightlist}{%
  \setlength{\itemsep}{0pt}\setlength{\parskip}{0pt}}
\setcounter{secnumdepth}{0}
% Redefines (sub)paragraphs to behave more like sections
\ifx\paragraph\undefined\else
\let\oldparagraph\paragraph
\renewcommand{\paragraph}[1]{\oldparagraph{#1}\mbox{}}
\fi
\ifx\subparagraph\undefined\else
\let\oldsubparagraph\subparagraph
\renewcommand{\subparagraph}[1]{\oldsubparagraph{#1}\mbox{}}
\fi

%%% Use protect on footnotes to avoid problems with footnotes in titles
\let\rmarkdownfootnote\footnote%
\def\footnote{\protect\rmarkdownfootnote}

%%% Change title format to be more compact
\usepackage{titling}

% Create subtitle command for use in maketitle
\newcommand{\subtitle}[1]{
  \posttitle{
    \begin{center}\large#1\end{center}
    }
}

\setlength{\droptitle}{-2em}

  \title{Assignment3\_BigData}
    \pretitle{\vspace{\droptitle}\centering\huge}
  \posttitle{\par}
    \author{Agus Setiyawan}
    \preauthor{\centering\large\emph}
  \postauthor{\par}
      \predate{\centering\large\emph}
  \postdate{\par}
    \date{4/15/2019}


\begin{document}
\maketitle

\begin{Shaded}
\begin{Highlighting}[]
\NormalTok{climate <-}\StringTok{ }\KeywordTok{read.csv}\NormalTok{ (}\StringTok{"climate_spending.csv"}\NormalTok{, }\DataTypeTok{header =} \OtherTok{TRUE}\NormalTok{)}
\KeywordTok{library}\NormalTok{(ggplot2)}
\end{Highlighting}
\end{Shaded}

\subsection{Read the climate data}\label{read-the-climate-data}

\begin{Shaded}
\begin{Highlighting}[]
\KeywordTok{summary}\NormalTok{(climate)}
\end{Highlighting}
\end{Shaded}

\begin{verbatim}
##            department      year       gcc_spending      
##  Agriculture    :18   Min.   :2000   Min.   :3.113e+07  
##  All Other      :18   1st Qu.:2004   1st Qu.:7.604e+07  
##  Commerce (NOAA):18   Median :2008   Median :1.552e+08  
##  Energy         :18   Mean   :2008   Mean   :3.465e+08  
##  Interior       :18   3rd Qu.:2013   3rd Qu.:3.209e+08  
##  NASA           :18   Max.   :2017   Max.   :1.676e+09  
##  NSF            :18
\end{verbatim}

\subsection{Make sure the data}\label{make-sure-the-data}

\begin{Shaded}
\begin{Highlighting}[]
\KeywordTok{attach}\NormalTok{ (climate)}
\KeywordTok{names}\NormalTok{ (climate)}
\end{Highlighting}
\end{Shaded}

\begin{verbatim}
## [1] "department"   "year"         "gcc_spending"
\end{verbatim}

\subsection{Plot the data}\label{plot-the-data}

ggplot by the year(x) and gcc\_spending (y) plotting by point:

\begin{Shaded}
\begin{Highlighting}[]
\KeywordTok{ggplot}\NormalTok{(climate, }\KeywordTok{aes}\NormalTok{(}\DataTypeTok{x =}\NormalTok{ year, }\DataTypeTok{y =}\NormalTok{gcc_spending, }\DataTypeTok{color =}\NormalTok{ department)) }\OperatorTok{+}
\StringTok{  }\KeywordTok{geom_point}\NormalTok{()}
\end{Highlighting}
\end{Shaded}

\includegraphics{Assignment3_BigData_files/figure-latex/plotting by point-1.pdf}

The gcc\_spending from NASA department has displayed that high value
compare between the other department, the data from 2000 to 2017 has
fluctiative and the highest showed between 2000 to 2003

ggplot by the year(x) and gcc\_spending (y) plotting by boxplot:

\begin{Shaded}
\begin{Highlighting}[]
\KeywordTok{ggplot}\NormalTok{(climate, }\KeywordTok{aes}\NormalTok{(}\DataTypeTok{x =}\NormalTok{ year, }\DataTypeTok{y =}\NormalTok{gcc_spending, }\DataTypeTok{color =}\NormalTok{ department)) }\OperatorTok{+}
\StringTok{  }\KeywordTok{geom_boxplot}\NormalTok{()}
\end{Highlighting}
\end{Shaded}

\includegraphics{Assignment3_BigData_files/figure-latex/plotting by boxplot-1.pdf}

The figure of the box plot showed that any big diffrences of the data on
the NASA Department on 2014. interestingly, that the small differences
of the data by Interior Department on 2011.

ggplot by the year(x) and gcc\_spending (y) plotting by line:

\begin{Shaded}
\begin{Highlighting}[]
\KeywordTok{ggplot}\NormalTok{(climate, }\KeywordTok{aes}\NormalTok{(}\DataTypeTok{x =}\NormalTok{ year, }\DataTypeTok{y =}\NormalTok{gcc_spending, }\DataTypeTok{color =}\NormalTok{ department)) }\OperatorTok{+}
\StringTok{  }\KeywordTok{geom_line}\NormalTok{()}
\end{Highlighting}
\end{Shaded}

\includegraphics{Assignment3_BigData_files/figure-latex/plotting by line-1.pdf}

Plotting the climate data sort by department and time series by year.
the figure showed that the gcc\_spending has fluctuative by the all
department, but NASA Department showed the high value compare to the
other Department.

\subsection{log by gcc\_spending}\label{log-by-gcc_spending}

\begin{Shaded}
\begin{Highlighting}[]
\NormalTok{df=climate}
\NormalTok{df}\OperatorTok{$}\NormalTok{lngcc_spending =}\StringTok{ }\KeywordTok{log}\NormalTok{(df}\OperatorTok{$}\NormalTok{gcc_spending)}
\end{Highlighting}
\end{Shaded}

\subsection{summary}\label{summary}

\begin{Shaded}
\begin{Highlighting}[]
\KeywordTok{summary}\NormalTok{(df)}
\end{Highlighting}
\end{Shaded}

\begin{verbatim}
##            department      year       gcc_spending       lngcc_spending 
##  Agriculture    :18   Min.   :2000   Min.   :3.113e+07   Min.   :17.25  
##  All Other      :18   1st Qu.:2004   1st Qu.:7.604e+07   1st Qu.:18.15  
##  Commerce (NOAA):18   Median :2008   Median :1.552e+08   Median :18.86  
##  Energy         :18   Mean   :2008   Mean   :3.465e+08   Mean   :19.02  
##  Interior       :18   3rd Qu.:2013   3rd Qu.:3.209e+08   3rd Qu.:19.59  
##  NASA           :18   Max.   :2017   Max.   :1.676e+09   Max.   :21.24  
##  NSF            :18
\end{verbatim}

\subsection{plotting log linear model by the department based on the
time series
(year):}\label{plotting-log-linear-model-by-the-department-based-on-the-time-series-year}

\begin{Shaded}
\begin{Highlighting}[]
\KeywordTok{ggplot}\NormalTok{(climate, }\KeywordTok{aes}\NormalTok{(}\DataTypeTok{x =}\NormalTok{ year, }\DataTypeTok{y =}\NormalTok{ gcc_spending, }\DataTypeTok{color =} \KeywordTok{factor}\NormalTok{(year)))}\OperatorTok{+}
\StringTok{  }\KeywordTok{geom_point}\NormalTok{() }\OperatorTok{+}\StringTok{ }\KeywordTok{scale_x_log10}\NormalTok{() }\OperatorTok{+}\StringTok{ }\KeywordTok{geom_smooth}\NormalTok{(}\DataTypeTok{method =} \StringTok{"lm"}\NormalTok{) }\OperatorTok{+}\StringTok{ }\KeywordTok{facet_wrap}\NormalTok{(}\OperatorTok{~}\NormalTok{department)}
\end{Highlighting}
\end{Shaded}

\includegraphics{Assignment3_BigData_files/figure-latex/plotting of the log gcc_spending linear model-1.pdf}
plotting by the linear and the year as factor, showed that the NASA
Department has a high value and fluctuative from 2000 - 2017.

\subsection{plotting log GLM by the department based on the time series
(year):}\label{plotting-log-glm-by-the-department-based-on-the-time-series-year}

\begin{Shaded}
\begin{Highlighting}[]
\KeywordTok{ggplot}\NormalTok{(climate, }\KeywordTok{aes}\NormalTok{(}\DataTypeTok{x =}\NormalTok{ year, }\DataTypeTok{y =}\NormalTok{ gcc_spending, }\DataTypeTok{color =} \KeywordTok{factor}\NormalTok{(year)))}\OperatorTok{+}
\StringTok{  }\KeywordTok{geom_point}\NormalTok{() }\OperatorTok{+}\StringTok{ }\KeywordTok{scale_x_log10}\NormalTok{() }\OperatorTok{+}\StringTok{ }\KeywordTok{geom_smooth}\NormalTok{(}\DataTypeTok{method =} \StringTok{"glm"}\NormalTok{) }\OperatorTok{+}\StringTok{ }\KeywordTok{facet_wrap}\NormalTok{ (}\OperatorTok{~}\NormalTok{department)}
\end{Highlighting}
\end{Shaded}

\includegraphics{Assignment3_BigData_files/figure-latex/plotting of the log gcc_spending glm-1.pdf}
plotting by the GLM has showed that the similar result with liner model
plotting.

\subsection{analysis Generalized Linear Model
(GLM)}\label{analysis-generalized-linear-model-glm}

\begin{Shaded}
\begin{Highlighting}[]
\NormalTok{glm_climate <-}\StringTok{ }\KeywordTok{glm}\NormalTok{(year }\OperatorTok{~}\StringTok{ }\NormalTok{gcc_spending }\OperatorTok{+}\StringTok{ }\NormalTok{department, }\DataTypeTok{family =}\NormalTok{ gaussian, }\DataTypeTok{data =}\NormalTok{ climate)}
\end{Highlighting}
\end{Shaded}

\subsection{summary of the GLM (Genaralized Linear
Model)}\label{summary-of-the-glm-genaralized-linear-model}

\begin{Shaded}
\begin{Highlighting}[]
\KeywordTok{summary}\NormalTok{ (glm_climate)}
\end{Highlighting}
\end{Shaded}

\begin{verbatim}
## 
## Call:
## glm(formula = year ~ gcc_spending + department, family = gaussian, 
##     data = climate)
## 
## Deviance Residuals: 
##      Min        1Q    Median        3Q       Max  
## -11.2060   -4.2741    0.4521    4.3017    9.1212  
## 
## Coefficients:
##                             Estimate Std. Error  t value Pr(>|t|)    
## (Intercept)                2.007e+03  1.348e+00 1488.739  < 2e-16 ***
## gcc_spending               1.698e-08  6.219e-09    2.730  0.00731 ** 
## departmentAll Other        7.815e-02  1.733e+00    0.045  0.96412    
## departmentCommerce (NOAA) -3.450e+00  2.145e+00   -1.608  0.11042    
## departmentEnergy          -1.596e+00  1.829e+00   -0.872  0.38473    
## departmentInterior         7.265e-01  1.753e+00    0.414  0.67938    
## departmentNASA            -2.277e+01  8.519e+00   -2.673  0.00859 ** 
## departmentNSF             -3.429e+00  2.141e+00   -1.602  0.11182    
## ---
## Signif. codes:  0 '***' 0.001 '**' 0.01 '*' 0.05 '.' 0.1 ' ' 1
## 
## (Dispersion parameter for gaussian family taken to be 27.03431)
## 
##     Null deviance: 3391.5  on 125  degrees of freedom
## Residual deviance: 3190.0  on 118  degrees of freedom
## AIC: 782.74
## 
## Number of Fisher Scoring iterations: 2
\end{verbatim}

the result of the GLM analysis showed that the significant by the NASA
Departmen and gcc\_spending, the start meaning is the significantly by
the year.

\subsection{Test Anova}\label{test-anova}

\begin{Shaded}
\begin{Highlighting}[]
\KeywordTok{anova}\NormalTok{(glm_climate)}
\end{Highlighting}
\end{Shaded}

\begin{verbatim}
## Analysis of Deviance Table
## 
## Model: gaussian, link: identity
## 
## Response: year
## 
## Terms added sequentially (first to last)
## 
## 
##              Df Deviance Resid. Df Resid. Dev
## NULL                           125     3391.5
## gcc_spending  1    5.318       124     3386.2
## department    6  196.134       118     3190.0
\end{verbatim}

the test anova showed that the Df value for the gcc\_spending higher
than Dpeartment by the time series data (year)

\subsection{Plot The GLM}\label{plot-the-glm}

\begin{Shaded}
\begin{Highlighting}[]
\KeywordTok{plot}\NormalTok{ (glm_climate)}
\end{Highlighting}
\end{Shaded}

\includegraphics{Assignment3_BigData_files/figure-latex/plot the GLM climate-1.pdf}
\includegraphics{Assignment3_BigData_files/figure-latex/plot the GLM climate-2.pdf}
\includegraphics{Assignment3_BigData_files/figure-latex/plot the GLM climate-3.pdf}
\includegraphics{Assignment3_BigData_files/figure-latex/plot the GLM climate-4.pdf}

Generalized linear model for the climate data was displayed that the
significantly to the NASA department by year. GLM analysis is describe
that the factor influencing to the variable.

\subsection{Read the Energy Data}\label{read-the-energy-data}

\section{Read the CSV data}\label{read-the-csv-data}

Read the CSV data from directory:

\begin{Shaded}
\begin{Highlighting}[]
\NormalTok{energy <-}\StringTok{ }\KeywordTok{read.csv}\NormalTok{ (}\StringTok{"energy_spending.csv"}\NormalTok{, }\DataTypeTok{header =} \OtherTok{TRUE}\NormalTok{)}
\end{Highlighting}
\end{Shaded}

\subsection{Summary}\label{summary-1}

\begin{Shaded}
\begin{Highlighting}[]
\KeywordTok{summary}\NormalTok{(energy)}
\end{Highlighting}
\end{Shaded}

\begin{verbatim}
##                               department       year     
##  Adv Sci Comp Res*                 : 22   Min.   :1997  
##  Atomic Energy Defense             : 22   1st Qu.:2002  
##  Basic Energy Sciences*            : 22   Median :2008  
##  Bio and Env Research*             : 22   Mean   :2008  
##  Energy Efficiency and Renew Energy: 22   3rd Qu.:2013  
##  Fossil Energy                     : 22   Max.   :2018  
##  (Other)                           :110                 
##  energy_spending    
##  Min.   :5.690e+07  
##  1st Qu.:4.762e+08  
##  Median :6.948e+08  
##  Mean   :1.456e+09  
##  3rd Qu.:1.357e+09  
##  Max.   :7.574e+09  
## 
\end{verbatim}

\subsection{Attach and Str the Data}\label{attach-and-str-the-data}

\begin{Shaded}
\begin{Highlighting}[]
\KeywordTok{attach}\NormalTok{ (energy)}
\end{Highlighting}
\end{Shaded}

\begin{verbatim}
## The following objects are masked from climate:
## 
##     department, year
\end{verbatim}

\begin{Shaded}
\begin{Highlighting}[]
\KeywordTok{str}\NormalTok{ (energy)}
\end{Highlighting}
\end{Shaded}

\begin{verbatim}
## 'data.frame':    242 obs. of  3 variables:
##  $ department     : Factor w/ 11 levels "Adv Sci Comp Res*",..: 11 1 3 4 7 8 10 5 9 6 ...
##  $ year           : int  1997 1997 1997 1997 1997 1997 1997 1997 1997 1997 ...
##  $ energy_spending: num  3.59e+09 2.17e+08 9.33e+08 5.51e+08 3.31e+08 ...
\end{verbatim}

\begin{Shaded}
\begin{Highlighting}[]
\KeywordTok{names}\NormalTok{ (energy)}
\end{Highlighting}
\end{Shaded}

\begin{verbatim}
## [1] "department"      "year"            "energy_spending"
\end{verbatim}

\subsection{Plot}\label{plot}

plotting the data by point ggplot:

\begin{Shaded}
\begin{Highlighting}[]
\KeywordTok{ggplot}\NormalTok{(energy, }\KeywordTok{aes}\NormalTok{(}\DataTypeTok{x =}\NormalTok{ year, }\DataTypeTok{y =}\NormalTok{energy_spending, }\DataTypeTok{color =}\NormalTok{ department)) }\OperatorTok{+}
\StringTok{   }\KeywordTok{geom_point}\NormalTok{()}
\end{Highlighting}
\end{Shaded}

\includegraphics{Assignment3_BigData_files/figure-latex/energy_plot-1.pdf}
The figure showed that the data for the energy spending has
significantly increase from 1997 until 2018 by the Atomic Energy Defense
Department. not only that Department, if we can see on the Adv. Sci Comp
Res Department showed that increase as well, but honestly start from
2010 has decreased.

plotting the data by boxplot ggplot:

\begin{Shaded}
\begin{Highlighting}[]
\KeywordTok{ggplot}\NormalTok{(energy, }\KeywordTok{aes}\NormalTok{(}\DataTypeTok{x =}\NormalTok{ year, }\DataTypeTok{y =}\NormalTok{energy_spending, }\DataTypeTok{color =}\NormalTok{ department)) }\OperatorTok{+}
\StringTok{   }\KeywordTok{geom_boxplot}\NormalTok{()}
\end{Highlighting}
\end{Shaded}

\includegraphics{Assignment3_BigData_files/figure-latex/plotting_boxplot-1.pdf}
The figure showed that any big differences on the Atomic Energy Defense
Department in the 2000 and also any differences by the Adv Sci Comp Res
Department in the 2016 compare to the other Department by time series
data from 1997 - 2018.

Plotting the data by line ggplot:

\begin{Shaded}
\begin{Highlighting}[]
\KeywordTok{ggplot}\NormalTok{(energy, }\KeywordTok{aes}\NormalTok{(}\DataTypeTok{x =}\NormalTok{ year, }\DataTypeTok{y =}\NormalTok{energy_spending, }\DataTypeTok{color =}\NormalTok{ department)) }\OperatorTok{+}
\StringTok{   }\KeywordTok{geom_line}\NormalTok{()}
\end{Highlighting}
\end{Shaded}

\includegraphics{Assignment3_BigData_files/figure-latex/plotting_line-1.pdf}
The figure showed that the Atomic Energy Defense Department has
fluctiative from 1997 until 2017 same as Atomic Energy Defense. but,
both of it has high value of energy spending compare to the other
Department. The data has recorded from 1997 until 2018.

\subsection{plotting log linear model by the department based on the
time series
(year):}\label{plotting-log-linear-model-by-the-department-based-on-the-time-series-year-1}

\begin{Shaded}
\begin{Highlighting}[]
\KeywordTok{ggplot}\NormalTok{(energy, }\KeywordTok{aes}\NormalTok{(}\DataTypeTok{x =}\NormalTok{ year, }\DataTypeTok{y =}\NormalTok{ energy_spending, }\DataTypeTok{color =} \KeywordTok{factor}\NormalTok{(year)))}\OperatorTok{+}
\StringTok{  }\KeywordTok{geom_point}\NormalTok{() }\OperatorTok{+}\StringTok{ }\KeywordTok{scale_x_log10}\NormalTok{() }\OperatorTok{+}\StringTok{ }\KeywordTok{geom_smooth}\NormalTok{(}\DataTypeTok{method =} \StringTok{"lm"}\NormalTok{) }\OperatorTok{+}\StringTok{ }\KeywordTok{facet_wrap}\NormalTok{(}\OperatorTok{~}\NormalTok{department)}
\end{Highlighting}
\end{Shaded}

\includegraphics{Assignment3_BigData_files/figure-latex/plotting_lm_ggplot-1.pdf}
The figure showed that the Atomic Energy Defense Department has a high
value of the energy spending over the year from 1997 - 2018. As same
thet the Office of Science R\&D Department has displayed that the value
is high over the year start from 1997 until 2018

\subsection{plotting log glm by the department based on the time series
(year):}\label{plotting-log-glm-by-the-department-based-on-the-time-series-year-1}

\begin{Shaded}
\begin{Highlighting}[]
\KeywordTok{ggplot}\NormalTok{(energy, }\KeywordTok{aes}\NormalTok{(}\DataTypeTok{x =}\NormalTok{ year, }\DataTypeTok{y =}\NormalTok{ energy_spending, }\DataTypeTok{color =} \KeywordTok{factor}\NormalTok{(year)))}\OperatorTok{+}
\StringTok{  }\KeywordTok{geom_point}\NormalTok{() }\OperatorTok{+}\StringTok{ }\KeywordTok{scale_x_log10}\NormalTok{() }\OperatorTok{+}\StringTok{ }\KeywordTok{geom_smooth}\NormalTok{(}\DataTypeTok{method =} \StringTok{"glm"}\NormalTok{) }\OperatorTok{+}\StringTok{ }\KeywordTok{facet_wrap}\NormalTok{ (}\OperatorTok{~}\NormalTok{department)}
\end{Highlighting}
\end{Shaded}

\includegraphics{Assignment3_BigData_files/figure-latex/plotting_glm_ggplot-1.pdf}
The figure displayed that not any difference with the lm ggplot as
displayed above, the value is high showed on the 2 Department (Atomic
Energy Defense and Office of Science R\&D Department)

\subsection{analysis Generalized Linear Model
(GLM)}\label{analysis-generalized-linear-model-glm-1}

\begin{Shaded}
\begin{Highlighting}[]
\NormalTok{glm_energy <-}\StringTok{ }\KeywordTok{glm}\NormalTok{(year }\OperatorTok{~}\StringTok{ }\NormalTok{energy_spending }\OperatorTok{+}\StringTok{ }\NormalTok{department, }\DataTypeTok{family =}\NormalTok{ gaussian, }\DataTypeTok{data =}\NormalTok{ energy)}
\end{Highlighting}
\end{Shaded}

\subsection{summary of the GLM (Genaralized Linear
Model)}\label{summary-of-the-glm-genaralized-linear-model-1}

\begin{Shaded}
\begin{Highlighting}[]
\KeywordTok{summary}\NormalTok{ (glm_energy)}
\end{Highlighting}
\end{Shaded}

\begin{verbatim}
## 
## Call:
## glm(formula = year ~ energy_spending + department, family = gaussian, 
##     data = energy)
## 
## Deviance Residuals: 
##      Min        1Q    Median        3Q       Max  
## -11.7463   -4.0891   -0.0892    4.2582   10.4473  
## 
## Coefficients:
##                                                Estimate Std. Error
## (Intercept)                                   2.004e+03  1.216e+00
## energy_spending                               8.968e-09  9.140e-10
## departmentAtomic Energy Defense              -4.227e+01  4.612e+00
## departmentBasic Energy Sciences*             -1.015e+01  1.945e+00
## departmentBio and Env Research*              -2.321e+00  1.664e+00
## departmentEnergy Efficiency and Renew Energy -6.038e+00  1.759e+00
## departmentFossil Energy                      -1.169e+00  1.652e+00
## departmentFusion Energy Sciences*            -7.648e-02  1.647e+00
## departmentHigh-Energy Physics*               -4.555e+00  1.712e+00
## departmentNuclear Energy                     -5.946e-01  1.649e+00
## departmentNuclear Physics*                   -1.384e+00  1.653e+00
## departmentOffice of Science R&D              -3.749e+01  4.161e+00
##                                               t value Pr(>|t|)    
## (Intercept)                                  1648.525  < 2e-16 ***
## energy_spending                                 9.812  < 2e-16 ***
## departmentAtomic Energy Defense                -9.165  < 2e-16 ***
## departmentBasic Energy Sciences*               -5.220    4e-07 ***
## departmentBio and Env Research*                -1.394 0.164521    
## departmentEnergy Efficiency and Renew Energy   -3.433 0.000707 ***
## departmentFossil Energy                        -0.708 0.479881    
## departmentFusion Energy Sciences*              -0.046 0.963015    
## departmentHigh-Energy Physics*                 -2.661 0.008330 ** 
## departmentNuclear Energy                       -0.361 0.718681    
## departmentNuclear Physics*                     -0.837 0.403358    
## departmentOffice of Science R&D                -9.011  < 2e-16 ***
## ---
## Signif. codes:  0 '***' 0.001 '**' 0.01 '*' 0.05 '.' 0.1 ' ' 1
## 
## (Dispersion parameter for gaussian family taken to be 29.85287)
## 
##     Null deviance: 9740.5  on 241  degrees of freedom
## Residual deviance: 6866.2  on 230  degrees of freedom
## AIC: 1522.4
## 
## Number of Fisher Scoring iterations: 2
\end{verbatim}

The analysis of GLM showed that the significantly factor to the variable
by the Department are Energy Defense Department, Basic Energy Science
Department, Energy Efficiency and Renew Energy Department, Office of
Science R\&D Department and High-Energy Physics Department. The Stars
meaning that any differences factor to the variable of the year.

\subsection{Anova Test}\label{anova-test}

\begin{Shaded}
\begin{Highlighting}[]
\KeywordTok{anova}\NormalTok{(glm_energy)}
\end{Highlighting}
\end{Shaded}

\begin{verbatim}
## Analysis of Deviance Table
## 
## Model: gaussian, link: identity
## 
## Response: year
## 
## Terms added sequentially (first to last)
## 
## 
##                 Df Deviance Resid. Df Resid. Dev
## NULL                              241     9740.5
## energy_spending  1   152.03       240     9588.5
## department      10  2722.31       230     6866.2
\end{verbatim}

The Anova test showed that the Df value is high to the Energy Spending
than Department, also the Residual Deviasi for 2 factor that the energy
spending has higher than Department value.

\section{Plot GLM}\label{plot-glm}

\begin{Shaded}
\begin{Highlighting}[]
\KeywordTok{plot}\NormalTok{ (glm_energy)}
\end{Highlighting}
\end{Shaded}

\includegraphics{Assignment3_BigData_files/figure-latex/plottig of energy data by glm-1.pdf}
\includegraphics{Assignment3_BigData_files/figure-latex/plottig of energy data by glm-2.pdf}
\includegraphics{Assignment3_BigData_files/figure-latex/plottig of energy data by glm-3.pdf}
\includegraphics{Assignment3_BigData_files/figure-latex/plottig of energy data by glm-4.pdf}

Generalized Linear Model for analysis is to estimate the factor
influence by the time series (year) from 1997 - 2018.

\subsection{Read the RD Data}\label{read-the-rd-data}

\subsection{Read the CSV data}\label{read-the-csv-data-1}

\begin{Shaded}
\begin{Highlighting}[]
\NormalTok{rd <-}\StringTok{ }\KeywordTok{read.csv}\NormalTok{ (}\StringTok{"fed_r_d_spending.csv"}\NormalTok{, }\DataTypeTok{header =} \OtherTok{TRUE}\NormalTok{)}
\KeywordTok{library}\NormalTok{(ggplot2)}
\end{Highlighting}
\end{Shaded}

\subsection{Summary}\label{summary-2}

\begin{Shaded}
\begin{Highlighting}[]
\KeywordTok{summary}\NormalTok{ (rd)}
\end{Highlighting}
\end{Shaded}

\begin{verbatim}
##    department       year        rd_budget         total_outlays      
##  DHS    : 42   Min.   :1976   Min.   :0.000e+00   Min.   :3.718e+11  
##  DOC    : 42   1st Qu.:1986   1st Qu.:9.020e+08   1st Qu.:9.904e+11  
##  DOD    : 42   Median :1996   Median :1.888e+09   Median :1.581e+12  
##  DOE    : 42   Mean   :1996   Mean   :1.035e+10   Mean   :1.880e+12  
##  DOT    : 42   3rd Qu.:2007   3rd Qu.:1.206e+10   3rd Qu.:2.729e+12  
##  EPA    : 42   Max.   :2017   Max.   :9.432e+10   Max.   :3.982e+12  
##  (Other):336                                                         
##  discretionary_outlays      gdp           
##  Min.   :1.756e+11     Min.   :1.790e+12  
##  1st Qu.:4.385e+11     1st Qu.:4.536e+12  
##  Median :5.460e+11     Median :8.230e+12  
##  Mean   :6.942e+11     Mean   :9.175e+12  
##  3rd Qu.:1.042e+12     3rd Qu.:1.432e+13  
##  Max.   :1.347e+12     Max.   :1.918e+13  
## 
\end{verbatim}

\subsection{Attach the data}\label{attach-the-data}

\begin{Shaded}
\begin{Highlighting}[]
\KeywordTok{attach}\NormalTok{ (rd)}
\end{Highlighting}
\end{Shaded}

\begin{verbatim}
## The following objects are masked from energy:
## 
##     department, year
\end{verbatim}

\begin{verbatim}
## The following objects are masked from climate:
## 
##     department, year
\end{verbatim}

\begin{Shaded}
\begin{Highlighting}[]
\KeywordTok{str}\NormalTok{ (rd)}
\end{Highlighting}
\end{Shaded}

\begin{verbatim}
## 'data.frame':    588 obs. of  6 variables:
##  $ department           : Factor w/ 14 levels "DHS","DOC","DOD",..: 3 9 4 7 10 11 13 8 5 6 ...
##  $ year                 : int  1976 1976 1976 1976 1976 1976 1976 1976 1976 1976 ...
##  $ rd_budget            : num  3.57e+10 1.25e+10 1.09e+10 9.23e+09 8.02e+09 ...
##  $ total_outlays        : num  3.72e+11 3.72e+11 3.72e+11 3.72e+11 3.72e+11 ...
##  $ discretionary_outlays: num  1.76e+11 1.76e+11 1.76e+11 1.76e+11 1.76e+11 ...
##  $ gdp                  : num  1.79e+12 1.79e+12 1.79e+12 1.79e+12 1.79e+12 ...
\end{verbatim}

\begin{Shaded}
\begin{Highlighting}[]
\KeywordTok{names}\NormalTok{ (rd)}
\end{Highlighting}
\end{Shaded}

\begin{verbatim}
## [1] "department"            "year"                  "rd_budget"            
## [4] "total_outlays"         "discretionary_outlays" "gdp"
\end{verbatim}

\subsection{The type of the Variable}\label{the-type-of-the-variable}

\begin{Shaded}
\begin{Highlighting}[]
\KeywordTok{typeof}\NormalTok{ (rd}\OperatorTok{$}\NormalTok{rd_budget)}
\end{Highlighting}
\end{Shaded}

\begin{verbatim}
## [1] "double"
\end{verbatim}

\begin{Shaded}
\begin{Highlighting}[]
\KeywordTok{typeof}\NormalTok{ (rd}\OperatorTok{$}\NormalTok{total_outlays)}
\end{Highlighting}
\end{Shaded}

\begin{verbatim}
## [1] "double"
\end{verbatim}

\begin{Shaded}
\begin{Highlighting}[]
\KeywordTok{typeof}\NormalTok{ (rd}\OperatorTok{$}\NormalTok{discretionary_outlays)}
\end{Highlighting}
\end{Shaded}

\begin{verbatim}
## [1] "double"
\end{verbatim}

\begin{Shaded}
\begin{Highlighting}[]
\KeywordTok{typeof}\NormalTok{ (rd}\OperatorTok{$}\NormalTok{gdp)}
\end{Highlighting}
\end{Shaded}

\begin{verbatim}
## [1] "double"
\end{verbatim}

\section{Plotting}\label{plotting}

Plotting the data rd (x = year, y = rd\_budget):

\begin{Shaded}
\begin{Highlighting}[]
\KeywordTok{ggplot}\NormalTok{(rd, }\KeywordTok{aes}\NormalTok{( }\DataTypeTok{x =}\NormalTok{ year, }\DataTypeTok{y =}\NormalTok{ rd_budget, }\DataTypeTok{color =}\NormalTok{ total_outlays)) }\OperatorTok{+}
\StringTok{  }\KeywordTok{geom_point}\NormalTok{()}
\end{Highlighting}
\end{Shaded}

\includegraphics{Assignment3_BigData_files/figure-latex/plotting the ggplot-1.pdf}
The figure displayed that the rd\_budget over time has increased from
1997 to 2018 by based on the total outylays inform that the total
outlays has incrreasing as well over the time.

Plotting the data rd (x = year, y = rd\_budget, based on the gdp):

\begin{Shaded}
\begin{Highlighting}[]
\KeywordTok{ggplot}\NormalTok{(rd, }\KeywordTok{aes}\NormalTok{( }\DataTypeTok{x =}\NormalTok{ year, }\DataTypeTok{y =}\NormalTok{ rd_budget, }\DataTypeTok{color =}\NormalTok{ gdp)) }\OperatorTok{+}
\StringTok{  }\KeywordTok{geom_point}\NormalTok{()}
\end{Highlighting}
\end{Shaded}

\includegraphics{Assignment3_BigData_files/figure-latex/plotting the ggplot 1-1.pdf}
The figure dispyaed that the rd\_budget has increased over time, as well
as gdp has increased over time.

Plottting the data (x = year, y = rd\_budget, based on the discreet
outlays):

\begin{Shaded}
\begin{Highlighting}[]
\KeywordTok{ggplot}\NormalTok{(rd, }\KeywordTok{aes}\NormalTok{( }\DataTypeTok{x =}\NormalTok{ year, }\DataTypeTok{y =}\NormalTok{ rd_budget, }\DataTypeTok{color =}\NormalTok{ discretionary_outlays)) }\OperatorTok{+}
\StringTok{  }\KeywordTok{geom_point}\NormalTok{()}
\end{Highlighting}
\end{Shaded}

\includegraphics{Assignment3_BigData_files/figure-latex/plotting the ggplot 2-1.pdf}
The figure showed that the rd\_budget has increased over time based on
the discretionary\_outlays.

Plotting the data (x = year, y = rd\_budget based on the Department):

\begin{Shaded}
\begin{Highlighting}[]
\KeywordTok{ggplot}\NormalTok{(rd, }\KeywordTok{aes}\NormalTok{( }\DataTypeTok{x =}\NormalTok{ year, }\DataTypeTok{y =}\NormalTok{ rd_budget, }\DataTypeTok{color =}\NormalTok{ department)) }\OperatorTok{+}
\StringTok{  }\KeywordTok{geom_point}\NormalTok{()}
\end{Highlighting}
\end{Shaded}

\includegraphics{Assignment3_BigData_files/figure-latex/plot the data by ggplot 3-1.pdf}
The figure displayed that based on the Department (DOD) has fluctuative
over the time (1997 - 2018) and has a high value of the budget compare
to the other Departement.

Plotting the data by linear model use the ggplot year and gdp:

\begin{Shaded}
\begin{Highlighting}[]
\KeywordTok{ggplot}\NormalTok{(rd, }\KeywordTok{aes}\NormalTok{(}\DataTypeTok{x =}\NormalTok{ year, }\DataTypeTok{y =}\NormalTok{gdp)) }\OperatorTok{+}
\StringTok{  }\KeywordTok{geom_point}\NormalTok{() }\OperatorTok{+}\StringTok{ }\KeywordTok{geom_smooth}\NormalTok{(}\DataTypeTok{method=}\StringTok{"lm"}\NormalTok{) }\OperatorTok{+}\StringTok{ }\KeywordTok{facet_wrap}\NormalTok{ (}\OperatorTok{~}\NormalTok{department)}
\end{Highlighting}
\end{Shaded}

\includegraphics{Assignment3_BigData_files/figure-latex/plotting the data by ggplot 4-1.pdf}
The figure displayed that based on the linear model showed the gdp has
increase over time time based on the all Department.

Plotting the data by linear model use the ggplot year and budget:

\begin{Shaded}
\begin{Highlighting}[]
\KeywordTok{ggplot}\NormalTok{(rd, }\KeywordTok{aes}\NormalTok{(}\DataTypeTok{x =}\NormalTok{ year, }\DataTypeTok{y =}\NormalTok{rd_budget)) }\OperatorTok{+}
\StringTok{  }\KeywordTok{geom_point}\NormalTok{() }\OperatorTok{+}\StringTok{ }\KeywordTok{geom_smooth}\NormalTok{(}\DataTypeTok{method=}\StringTok{"lm"}\NormalTok{) }\OperatorTok{+}\StringTok{ }\KeywordTok{facet_wrap}\NormalTok{ (}\OperatorTok{~}\NormalTok{department)}
\end{Highlighting}
\end{Shaded}

\includegraphics{Assignment3_BigData_files/figure-latex/plotting the data by ggplot 5-1.pdf}
The figure showed that DOD department has fluctuative distribution over
the time, we can see that the increase of the budget has high
significant increase over the time by linear model, also we can see the
high slope. NHS and NIH department are showed that increase significant
compare than the other department, but the DOD Department has a high
significantly increase over time.

Plotting the data by linear model use the ggplot year and total outlays:

\begin{Shaded}
\begin{Highlighting}[]
\KeywordTok{ggplot}\NormalTok{(rd, }\KeywordTok{aes}\NormalTok{(}\DataTypeTok{x =}\NormalTok{ year, }\DataTypeTok{y =}\NormalTok{total_outlays)) }\OperatorTok{+}
\StringTok{  }\KeywordTok{geom_point}\NormalTok{() }\OperatorTok{+}\StringTok{ }\KeywordTok{geom_smooth}\NormalTok{(}\DataTypeTok{method=}\StringTok{"lm"}\NormalTok{) }\OperatorTok{+}\StringTok{ }\KeywordTok{facet_wrap}\NormalTok{ (}\OperatorTok{~}\NormalTok{department)}
\end{Highlighting}
\end{Shaded}

\includegraphics{Assignment3_BigData_files/figure-latex/plottimg the data by ggplot 6-1.pdf}
The figure displayed that based on the total outlays over the time has
increase, and the data fit to the line by linear model, it means that
the data over time has significantly increase on the all Department over
the time.

Plotting the data by linear model use the ggplot year and discretionary
outlays:

\begin{Shaded}
\begin{Highlighting}[]
\KeywordTok{ggplot}\NormalTok{(rd, }\KeywordTok{aes}\NormalTok{(}\DataTypeTok{x =}\NormalTok{ year, }\DataTypeTok{y =}\NormalTok{discretionary_outlays)) }\OperatorTok{+}
\StringTok{  }\KeywordTok{geom_point}\NormalTok{() }\OperatorTok{+}\StringTok{ }\KeywordTok{geom_smooth}\NormalTok{(}\DataTypeTok{method=}\StringTok{"lm"}\NormalTok{) }\OperatorTok{+}\StringTok{ }\KeywordTok{facet_wrap}\NormalTok{ (}\OperatorTok{~}\NormalTok{department)}
\end{Highlighting}
\end{Shaded}

\includegraphics{Assignment3_BigData_files/figure-latex/plotting the data by ggplot 7-1.pdf}
The figure showed that over the time based on the all Department by the
discretionary outlays has increased and the linear model has fitted to
the line, it means that the data has increased significantly.

Plotting the data by glm on the ggplot over the time:

\begin{Shaded}
\begin{Highlighting}[]
\KeywordTok{ggplot}\NormalTok{(rd, }\KeywordTok{aes}\NormalTok{(}\DataTypeTok{x =}\NormalTok{ year, }\DataTypeTok{y =}\NormalTok{ total_outlays, }\DataTypeTok{color =}\NormalTok{ gdp))}\OperatorTok{+}
\StringTok{  }\KeywordTok{geom_point}\NormalTok{() }\OperatorTok{+}\StringTok{ }\KeywordTok{scale_x_log10}\NormalTok{() }\OperatorTok{+}\StringTok{ }\KeywordTok{geom_smooth}\NormalTok{(}\DataTypeTok{method =} \StringTok{"glm"}\NormalTok{)}
\end{Highlighting}
\end{Shaded}

\includegraphics{Assignment3_BigData_files/figure-latex/plotting the data by ggplot-glm 1-1.pdf}
The figure dispyaed that the total overlays data over the time has
increased. The model has fitted to the data, it means that the data has
increased significantly over the time.

Plotting the data by glm on the ggplot over the time:

\begin{Shaded}
\begin{Highlighting}[]
\KeywordTok{ggplot}\NormalTok{(rd, }\KeywordTok{aes}\NormalTok{(}\DataTypeTok{x =}\NormalTok{ year, }\DataTypeTok{y =}\NormalTok{ total_outlays, }\DataTypeTok{color =}\NormalTok{ rd_budget))}\OperatorTok{+}
\StringTok{  }\KeywordTok{geom_point}\NormalTok{() }\OperatorTok{+}\StringTok{ }\KeywordTok{scale_x_log10}\NormalTok{() }\OperatorTok{+}\StringTok{ }\KeywordTok{geom_smooth}\NormalTok{(}\DataTypeTok{method =} \StringTok{"glm"}\NormalTok{)}
\end{Highlighting}
\end{Shaded}

\includegraphics{Assignment3_BigData_files/figure-latex/plotting the data by ggplot - glm 2-1.pdf}
The figure showed that the rd budget has a low over time based on the
color, but the total outlays has increased over the time. The data
fitted to the line by glm, it means that the data has increased
significantly over the time.

\subsection{GLM (Generalized Linear Model)
Analysis}\label{glm-generalized-linear-model-analysis}

The GLM analysis is to assess what is the factor influence by the total
outlays: m

\begin{Shaded}
\begin{Highlighting}[]
\NormalTok{glm_rd <-}\StringTok{ }\KeywordTok{glm}\NormalTok{(total_outlays }\OperatorTok{~}\StringTok{ }\NormalTok{year }\OperatorTok{+}\StringTok{ }\NormalTok{rd_budget }\OperatorTok{+}\StringTok{ }\NormalTok{department }\OperatorTok{+}\StringTok{ }\NormalTok{gdp }\OperatorTok{+}\StringTok{ }\NormalTok{discretionary_outlays, }\DataTypeTok{family =}\NormalTok{ gaussian, }\DataTypeTok{data =}\NormalTok{ rd)}
\end{Highlighting}
\end{Shaded}

\subsection{Summary}\label{summary-3}

\begin{Shaded}
\begin{Highlighting}[]
\KeywordTok{summary}\NormalTok{ (glm_rd)}
\end{Highlighting}
\end{Shaded}

\begin{verbatim}
## 
## Call:
## glm(formula = total_outlays ~ year + rd_budget + department + 
##     gdp + discretionary_outlays, family = gaussian, data = rd)
## 
## Deviance Residuals: 
##        Min          1Q      Median          3Q         Max  
## -2.647e+11  -2.075e+10   1.154e+10   4.934e+10   3.322e+11  
## 
## Coefficients:
##                         Estimate Std. Error t value Pr(>|t|)    
## (Intercept)            3.142e+13  5.447e+12   5.769 1.31e-08 ***
## year                  -1.599e+10  2.758e+09  -5.800 1.10e-08 ***
## rd_budget             -2.598e+00  7.958e-01  -3.265  0.00116 ** 
## departmentDOC          2.215e+09  2.324e+10   0.095  0.92411    
## departmentDOD          1.671e+11  5.620e+10   2.973  0.00308 ** 
## departmentDOE          2.989e+10  2.497e+10   1.197  0.23182    
## departmentDOT          1.400e+09  2.324e+10   0.060  0.95198    
## departmentEPA          9.650e+08  2.323e+10   0.042  0.96688    
## departmentHHS          5.694e+10  2.905e+10   1.960  0.05047 .  
## departmentInterior     1.355e+09  2.323e+10   0.058  0.95351    
## departmentNASA         3.056e+10  2.505e+10   1.220  0.22297    
## departmentNIH          5.388e+10  2.850e+10   1.891  0.05917 .  
## departmentNSF          9.508e+09  2.341e+10   0.406  0.68482    
## departmentOther        2.899e+09  2.325e+10   0.125  0.90082    
## departmentUSDA         5.201e+09  2.329e+10   0.223  0.82335    
## departmentVA           9.220e+08  2.323e+10   0.040  0.96836    
## gdp                    1.489e-01  7.184e-03  20.722  < 2e-16 ***
## discretionary_outlays  1.472e+00  4.689e-02  31.396  < 2e-16 ***
## ---
## Signif. codes:  0 '***' 0.001 '**' 0.01 '*' 0.05 '.' 0.1 ' ' 1
## 
## (Dispersion parameter for gaussian family taken to be 1.133344e+22)
## 
##     Null deviance: 7.1508e+26  on 587  degrees of freedom
## Residual deviance: 6.4601e+24  on 570  degrees of freedom
## AIC: 31548
## 
## Number of Fisher Scoring iterations: 2
\end{verbatim}

The result showed that the factor influences significantly are year, rd
budget, department DOC, gdp and discreetionary outlays. The start
meaning that has significantly influences to the variable.

\subsection{Plotting the GLM}\label{plotting-the-glm}

plotting the GLM result :

\begin{Shaded}
\begin{Highlighting}[]
\KeywordTok{plot}\NormalTok{(glm_rd)}
\end{Highlighting}
\end{Shaded}

\includegraphics{Assignment3_BigData_files/figure-latex/glm plotting the rd data-1.pdf}
\includegraphics{Assignment3_BigData_files/figure-latex/glm plotting the rd data-2.pdf}
\includegraphics{Assignment3_BigData_files/figure-latex/glm plotting the rd data-3.pdf}
\includegraphics{Assignment3_BigData_files/figure-latex/glm plotting the rd data-4.pdf}

\subsection{Anova Test}\label{anova-test-1}

\begin{Shaded}
\begin{Highlighting}[]
\KeywordTok{anova}\NormalTok{(glm_rd)}
\end{Highlighting}
\end{Shaded}

\begin{verbatim}
## Analysis of Deviance Table
## 
## Model: gaussian, link: identity
## 
## Response: total_outlays
## 
## Terms added sequentially (first to last)
## 
## 
##                       Df   Deviance Resid. Df Resid. Dev
## NULL                                      587 7.1508e+26
## year                   1 6.7994e+26       586 3.5138e+25
## rd_budget              1 2.9410e+20       585 3.5138e+25
## department            13 2.6820e+21       572 3.5136e+25
## gdp                    1 1.7504e+25       571 1.7631e+25
## discretionary_outlays  1 1.1171e+25       570 6.4601e+24
\end{verbatim}

The analysis of anova showed that the year has a high value for the Df
and the residuls deviance showed that gdp has low residuals. The high
residuals Deviance value is discretionary outlays.


\end{document}
